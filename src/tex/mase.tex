\documentclass[9pt,twocolumn]{article}





\usepackage{graphicx}
\usepackage{natbib}
\bibliographystyle{unsrtnat}
\usepackage{lmodern}
\usepackage{amssymb,amsmath}
\usepackage{ifxetex,ifluatex}
\usepackage{fixltx2e} % provides \textsubscript
\ifnum 0\ifxetex 1\fi\ifluatex 1\fi=0 % if pdftex
  \usepackage[T1]{fontenc}
  \usepackage[utf8]{inputenc}
\else % if luatex or xelatex
  \ifxetex
    \usepackage{mathspec}
    \usepackage{xltxtra,xunicode}
  \else
    \usepackage{fontspec}
  \fi
  \defaultfontfeatures{Mapping=tex-text,Scale=MatchLowercase}
  \newcommand{\euro}{€}
  \fi
  \usepackage{wrapfig}
% use upquote if available, for straight quotes in verbatim environments
\IfFileExists{upquote.sty}{\usepackage{upquote}}{}
% use microtype if available
\IfFileExists{microtype.sty}{%
\usepackage{microtype}
\UseMicrotypeSet[protrusion]{basicmath} % disable protrusion for tt fonts
}{}
\ifxetex
  \usepackage[setpagesize=false, % page size defined by xetex
              unicode=false, % unicode breaks when used with xetex
              xetex]{hyperref}
\else
  \usepackage[unicode=true]{hyperref}
  \fi

%  \usepackage[T1]{fontenc}
\renewcommand{\rmdefault}{ptm}
\usepackage{fourier}

\usepackage[scaled=0.875]{helvet} % ss
\renewcommand{\ttdefault}{pcr} %tt

\usepackage[dvipsnames]{xcolor}
\hypersetup{breaklinks=true,
            bookmarks=true,
            pdfauthor={},
            pdftitle={},
            colorlinks=true,
            citecolor=blue,
            urlcolor=blue,
            linkcolor=magenta,
            pdfborder={0 0 0}}
\urlstyle{same}  % don't use monospace font for urls
\setlength{\parindent}{0pt}
\setlength{\parskip}{6pt plus 2pt minus 1pt}
\setlength{\emergencystretch}{3em}  % prevent overfull lines

\usepackage{lastpage}
\usepackage{fancyhdr}
\pagestyle{fancy}

\rhead{} \chead{}\lhead{}
\cfoot{}

\fancyfoot[LE,RO]{page~\thepage~of~\pageref{LastPage}}
\fancyfoot[LO,RE]{\rotatebox[origin=c]{180}{\copyright} 2015, Tim Menzies, sort of.}



  \renewcommand{\footrulewidth}{0.4pt}

\date{}
%\usepackage{times}
\usepackage[margin=0.65in,landscape]{geometry}

\makeatletter
\newcommand{\verbatimfont}[1]{\def\verbatim@font{#1}}%
\makeatother
\usepackage{fancyvrb}


\usepackage[shortlabels]{enumitem}
\setlist{nosep}

\DefineVerbatimEnvironment%
  {MyVerbatim}{Verbatim}
  {numbers=right,numbersep=2mm,firstnumber=last,stepnumber=1,xleftmargin=12pt,xrightmargin=24pt,
   frame=lines,framerule=0.1mm,fontsize=\scriptsize,rulecolor=\color{Gray}}

  \definecolor{britishracinggreen}{rgb}{0.0, 0.26, 0.15}
  \definecolor{bulgarianrose}{rgb}{0.28, 0.02, 0.03}
  \definecolor{coolblack}{rgb}{0.0, 0.18, 0.39}
  \definecolor{mygreen}{rgb}{0,0.6,0}
  \definecolor{mymauve}{rgb}{0.58,0,0.82}
  
  \usepackage{listings}

  \colorlet{shadecolor}{gray!8}
  \lstset{% general command to set parameter(s)
    language=Python,
    firstnumber=last,
    xleftmargin=6pt,
    xrightmargin=12pt,
    numberblanklines=false,
    backgroundcolor = \color{shadecolor},
    numbers=right, stepnumber=1, numberstyle=\tiny, numbersep=1pt,
    basicstyle=\linespread{0.90}\scriptsize\ttfamily, % print whole listing small
    keywordstyle=\color{coolblack}\bfseries,
    emphstyle=\ttb\color{bulgarianrose},    % Custom highlighting style
    stringstyle=\color{britishracinggreen},
    commentstyle=\color{red},
    stringstyle=\color{mygreen},
    %frame=tb,framerule=0.5pt,rulecolor=\color{Gray},
      showstringspaces=false,
    } % no special string spaces

 \newcommand{\said}[1]{\citet*{#1}}


 \hypersetup{linkcolor=black}
\setcounter{tocdepth}{2}

\newcommand{\eq}[1]{Equation~\ref{eq:#1}}
\newcommand{\bi}{\begin{itemize}}
\newcommand{\ei}{\end{itemize}}
\newcommand{\be}{\begin{enumerate}}
\newcommand{\ee}{\end{enumerate}}
\newcommand{\tion}[1]{\textsection\ref{sec:#1}}
\newcommand{\fig}[1]{Figure~\ref{fig:#1}}

\setlength{\columnsep}{10mm}
\setlength{\columnseprule}{0.1pt}

\renewcommand{\headwidth}{10in}
%% \makeatletter
%%     \def\headrule{{\if@fancyplain\let\headrulewidth\plainheadrulewidth\fi
%%         \hrule\@height\headrulewidth\@width 7in \vskip-\headrulewidth}}
%% \makeatother

%% \makeatletter
%%     \def\footrule{{\if@fancyplain\let\headrulewidth\plainheadrulewidth\fi
%%         \hrule\@height\headrulewidth\@width 7in \vskip-\headrulewidth}}
%% \makeatother

\usepackage{tocloft}

\setlength{\cftbeforesecskip}{6pt}

\usepackage{moresize}



\newcommand{\parting}[2]{\newpage
\onecolumn
\clearpage
\fancyhead[LE,RO]{}
\fancyhead[LO,RE]{}
\vspace*{\fill}
\begin{center}
  \addcontentsline{toc}{section}{#1~~~#2}
  \begin{minipage}{.6\textwidth}
    \begin{center}
      { \fontsize{450}{450} \selectfont {\bf #1}}\\~\\ \LARGE{(#2)}\end{center}
\end{minipage}
\end{center}
\vfill % equivalent to \vspace{\fill}
\clearpage
\fancyhead[LE,RO]{\slshape \rightmark}
\fancyhead[LO,RE]{\slshape \leftmark}
\twocolumn
}
  
  
\begin{document} 

%\renewcommand{\verbatim@font}{\ttfamily\small}

\onecolumn
\title{
  {\bf \HUGE{Evil Code for Wicked Problems, part 4}}\\~\\~\\
    \fcolorbox{red}{white}{\includegraphics[width=6.5in]{img/herGoggles.png}}\\
    \Large{A Research Programmer's Guide to World Domination-- in Python.}\\
    \Large{a.k.a. {\em Lecture Notes, Automated
  SE}, CS, NC State, Fall'15}}
\author{by Tim Menzies \\\#attentionDeficitSquirrel\\Download:
  see {\tt book.pdf} at https://github.com/txt/evil\\This version: \today}
   

\maketitle
\thispagestyle{empty}

\clearpage
\small
\twocolumn

\pagestyle{fancy}



 \subsection*{About this book} This book is a  ``how to'' guide about model-based reasoning using
   data mining and search-based tools (with examples taken from software engineering).
   It is intended for graduate  students taking
  a one semester subject in advanced programming methods as
  well as researchers developing the next generation
  of model-based reasoning tools.  

  Using Python 2.7, the book builds (from the ground up) numerous
  tiny tools that can tame seemingly complex
  tasks. The combined toolkit, called RINSE, offers four kinds of functionality:
  \be
  \item
    It {\em \underline{r}epresents}  models using domain-specific languages;
    \item It supports
    {\em \underline{in}ference} across the multiple goals of those models using multi-objective optimization.
    \item It shows how to succinctly {\em \underline{s}ummarize} that inference  using data miners;
    \item It has  many tools for the
   {\em \underline{e}valuation} of different inference methods. 
   \ee
   
   RINSE is a not some shiny  end-user click-and-point GUI package.
   Rather, it is a starter-kit that demonstrates an novel  model-based approach to problem solving where programmers
   mix and match and extend data miners and multi-objective optimizers.

   RINSE was written using the mantra ``less is more''. Whenever it was found that small parts of the
   the code handled most 
   of the functionality, then the extra functionality was ejected. This resulted in a (very) small code base
   which can be readily browsed, learned, taught, and changed.


   \vfill


   \subsection*{Content Advisory }
\begin{wrapfigure}{r}{1.5in}
  \includegraphics[width=1.5in]{img/shark.jpg}
  \end{wrapfigure}
This book contains strong language, weakly typed (and tapped with glee).

This book may contain excessive or gratutious fun--
as well as ideas that some readers may (or may not) find disturbing.
This book does not necessarily 
believed or endorse those ideas- but plays with them anyway
(and asks you to do the same).

This book may include heresies, not suitable for anyone  who believes in established wisdom, without
adequate experimentation. It is
intended for mature audiences only; i.e.  those
old enough to know there is much left to know. 

This book may (or may not) contain peanuts or tree nut products.

Batteries not included.

\newpage

\tableofcontents

\newpage

   
\subsection*{Source Code Availability and Copyleft}
To download the RINSE code, see http://github.com/txt/mase.
The software associated with this book
is free and unencumbered and released into the public domain. 

Anyone is free to copy, modify, publish, use, compile, sell, or
distribute this software, either in source code form or as a compiled
binary, for any purpose, commercial or non-commercial, and by any
means.

In jurisdictions that recognize copyright laws, the author or authors
of this software dedicate any and all copyright interest in the
software to the public domain. We make this dedication for the benefit
of the public at large and to the detriment of our heirs and
successors. We intend this dedication to be an overt act of
relinquishment in perpetuity of all present and future rights to this
software under copyright law.

THE SOFTWARE IS PROVIDED "AS IS", WITHOUT WARRANTY OF ANY KIND,
EXPRESS OR IMPLIED, INCLUDING BUT NOT LIMITED TO THE WARRANTIES OF
MERCHANTABILITY, FITNESS FOR A PARTICULAR PURPOSE AND NONINFRINGEMENT.
IN NO EVENT SHALL THE AUTHORS BE LIABLE FOR ANY CLAIM, DAMAGES OR
OTHER LIABILITY, WHETHER IN AN ACTION OF CONTRACT, TORT OR OTHERWISE,
ARISING FROM, OUT OF OR IN CONNECTION WITH THE SOFTWARE OR THE USE OR
OTHER DEALINGS IN THE SOFTWARE.

For more information, please refer to http://unlicense.org

\vfill

   \subsection*{About the Author}

\begin{wrapfigure}{r}{1.5in}
\includegraphics[width=1.5in]{img/tim.png}
\end{wrapfigure}
Tim Menzies (Ph.D., UNSW, 1995, http://menzies.us) is a full Professor in CS at North Carolina State University where he teaches software engineering and automated software engineering. His research relates to synergies between human and artificial intelligence, with particular application to data mining for software engineering.

In his career, he has been a lead researcher on projects for NSF, NIJ, DoD, NASA, USDA, as well as joint research work with private companies.
He is the author of over 230 referred publications; and is one of the 100 most cited authors in software engineering out of over 80,000 researchers.

Prof. Menzies is an associate editor of IEEE
Transactions on Software Engineering, Empirical
Software Engineering and the Automated Software
Engineering Journal. His community service includes
co-founder of the PROMISE project (storing data for repeatable SE
experiments);
co-program chair for the 2012 conference on Automated SE and the 2015
New Ideas and Emerging Research track at the International Conference on SE;
and 
co-general chair for 2016 International Conference on Software Maintenance and Evolution.

Prof. Menzies can be contacted at \verb!tim.menzies@gmail.com!.




\newpage




%\renewcommand\contentsname{}



\parting{A}{An Introduction}
\section{Welcome to the Evil Plan}\label{welcome-to-the-evil-plan}

\textbf{``The world is a dangerous place to live, not because of the
people who are evil, but because of the people who don't do anything
about it.''} - Albert Einstein

The evil plan (by programmers) to take over the world is progressing
nicely. Certain parts of that plan were initially somewhat undefined.
However, given recent results, this book can now fill in the missing
details from part4 of that plan.

But first, a little history. As all programmers know, the initial parts
of the plan were completed years ago. Part one was was programmers to
adopt a meek and mild persona (possibly even boring and dull).

Part two was, under the guise of that persona, ingratiated ourselves to
government and indistrial agenices (education, mining, manufacturing,
etc etc). Once there, make our work essential to their day to day
opertion. Looking around the world today, it it is plain to see that
part two was very successful.

After that, part three was to make much more material available for our
inspection and manipluation. To this end, the entire planet was enclosed
a digital network- thus giving us unprecendented access to petabytes of
sensors and effectors. Also, by carefully seeding a few promienet
examples of successful programmers (Bill Gates, Steve Jobs, Mark
Zuckerburg), we convinced a lot of people to write lots of little tools,
each of which represent or control some thing, somewhere.

Part four was a little tricky but, as shown in this book, it turned out
not to be too hard. Having access to many models and much data can be
overwhelming-- unless some GREAT SECRET can be used to significantly
simply all that information. For the longest time, that GREAT SECRET was
unknown. However, recent advances have revealed the GREAT that SECRET--
if we describe something in \emph{N} dimensions, then there is usually a
much smaller set of \emph{M} dimensions that contain most of the signal.
The GREAT SECRET is that is it very easy (and fast) to find, then
exploit, those few number of \emph{M} dimensions.

\begin{wrapfigure}{r}{1.3in}
\includegraphics[width=1.3in]{img/evillaugh.jpg}
\end{wrapfigure}

With those controllers in hand, we are now free to move to part five;
i.e.~taking over the world. In fact, the truly evil part of this work is
that know you know you have the power to change the world. Which also
means (evil laugh) you have the guilt if you do not use that power to
right the wrongs of the world. So welcome to a lifetime of discontent
(punctuated by the occasionaly, perhaps fleerting, truimphs) as you
struggle to solve a very large number of pressing problems facing
humanity.

'Nough said. Good luck taking over the world. Remember: if you don't try
then you won't be able to sleep at night, ever again (final evil laugh).

\newpage
\subsection{Research Programming}\label{research-programming}

Silliness aside, this book is about how to be a \emph{research
programmer}. Research programmer's understand the world by:

\begin{itemize}
\itemsep1pt\parskip0pt\parsep0pt
\item
  Codify out current understanding of ``it'' into a model.
\item
  Reasoning about the model.
\end{itemize}

We take this term ``research programmer'' from Ph.D.~Steve Guao's 2012
dissertation.

\subsubsection{Challenges with Research
Programming}\label{challenges-with-research-programming}

Research programming sounds simple, right? Well, there's a catch
(actually, there are several catches).

Firstly, models have to be written and it can be quite a task to create
and validate a model of some complex phenomenon.

see also list in sbse14

Secondly, many models related to \emph{wicked problems};
i.e.\textasciitilde{}problems for which there is no clear best solution.
Tittel XXXWorse still, some models relate to \_wicked there is final
matter of the \emph{goals} that humans want to achieve with those
models. When those goals are contradictory (which happens, all too
often), then our model-based tools must negotiate complex trade offs
between different possibilities.

Thirdly, if wicked problems were not eough, there is also the issue of
uncertainty. Many real world models contain large areas of uncertainty,
especially if that model relates to something that humans have only been
studying for a few decades.

Fourthly, even if you are still not worried about the effectiveness of
reserach problem, consider the complexity of real-world phenomonem. Many
of these models are so complex that we cannot predict what happens when
the parts of that model interact.

Sounds simple, right? Well, there's a catch. Many models related to
\emph{wicked problems}; i.e.~problems for which there is no clear best
solution. Tittel XXXWorse still, some models relate to \_wicked there is
final matter of the \emph{goals} that humans want to achieve with those
models. When those goals are contradictory (which happens, all too
often), then our model-based tools must negotiate complex trade offs
between different possibilities.

If wicked problems were not eough, there is also the issue of
uncertainty. Many real world models contain large areas of uncertainty,
especially if that model relates to something that humans have only been
studying for a few decades.

And if you are still not worried about the effectiveness of reserach
problem, consider the complexity of real-world phenomonem. Many of these
models are so complex that we cannot predict what happens when the parts
of that model interact.

\subsubsection{Parts}\label{parts}

\begin{itemize}
\itemsep1pt\parskip0pt\parsep0pt
\item
  Domain specifc langauges (representation)
\item
  execution (nuktu-objective ootiization)
\item
  evaluation (statistical methods for experimental sciencetists in SE)
\item
  Philophsopy (about what it means to know, and to doubt)
\end{itemize}

\subsubsection{Implications for Software
Engineering}\label{implications-for-software-engineering}

Note that research programming changes the nature and focus and role of
21st century software engineering:

\begin{itemize}
\itemsep1pt\parskip0pt\parsep0pt
\item
  Traditionally, software engineering is about services that meet
  requirements.
\item
  But with research programming, software engineering is less about
  service than about search. Research programming's goal is the
  discovery of interesting features in existing models (or perhaps even
  the evolution of entirely new kinds of models).
\end{itemize}

For example, old-fashioned software engineerings might explore small
things like strings or ``hello world''. But with research programmers
explore \textbf{BIG} things like String Theory or ``hello world model of
climate change and economic impacts''.



\subsection*{The GREAT SECRET}

\subsection*{Example}

brook's law.  DSL in python of CM. data mining.

\parting{B}{Before we begin}

\section{Before we Begin}\label{before-we-begin}

Our goals are lofty- introducing a new paradigm that combines data
mining with multi-objective optimization. And doing so in such a way
that even novices can understand, use, and adapt these tools for a large
range of new tasks.

But before we can start all that, we have to handle some preliminaries.
All artists, and programmers, should start out as apprentices. If we
were painters and this was Renaissance Italy, us apprentices would spend
decades study the ways of the masters, all the while preparing the
wooden panels for painting; agrinding and mixing pigments; drawing
preliminary sketches, copying paintings, and casting sculptures. It was
a good system that gave us the Michelangelo and Da Vinci who, in turn,
gave us the roof of the Sistine Chapel and the Mona Lisa.

In terms of this book, us apprentices first have to become effective
Python programmers. The rest of this chapter offers:

\begin{itemize}
\itemsep1pt\parskip0pt\parsep0pt
\item
  Some notes on useful web-based programming tools
\item
  Some pointers on learning Python
\item
  Some start-up exercises to test if you have an effective Python
  programming environment.
\end{itemize}

\subsection{Useful On-Line Tools}\label{useful-on-line-tools}

\subsubsection{Stackoverflow}\label{stackoverflow}

To find answers to nearly any question you'll ever want to ask about
Python, go browse:

\begin{lstlisting}
 http://stackoverflow.com/questions/tagged/python
\end{lstlisting}

\subsubsection{Github}\label{github}

All programmers should use off-site backup for their work. All
programmers working in teams should store their code in repositories
that let them fork a branch, work separately, then check back their
changes into the main trunk.

There are many freely-available repository tools. Github is one such
service that supports the \texttt{git} repository tool. Github has some
special advantages:

\begin{itemize}
\itemsep1pt\parskip0pt\parsep0pt
\item
  It is the center of vast social network of programmers;
\item
  Github support serving static web sites straight from your Github
  repo.
\item
  Many other services offer close integration with Github (e.g.~the
  Cloud9 tool discussed below).
\end{itemize}

For more information, go to:

\begin{lstlisting}
 http://github.com
\end{lstlisting}

For Linux/Unix/Mac users, I add the following tip. In each of your
repository directories, add a \texttt{Makefile} with the following
contents.

\begin{lstlisting}
typo:   ready
        @- git status
        @- git commit -am "saving"
        @- git push origin master # update as needed

commit: ready
        @- git status
        @- git commit -a
        @- git push origin master

update: ready
        @- git pull origin master

status: ready
        @- git status

ready:
        @git config --global credential.helper cache
        @git config credential.helper \
             'cache --timeout=3600'

timm:  # <== change to your name
        @git config --global user.name "Tim Menzies"
        @git config --global user.email \
                               tim.menzies@gmail.com
\end{lstlisting}

This \texttt{Makefile} implements some handy shortcuts:

\begin{itemize}
\itemsep1pt\parskip0pt\parsep0pt
\item
  \texttt{make\ typo} is a quick safety save-- do this many times per
  day;
\item
  \texttt{make\ commit} is for making commented commits-- use this to
  comment any improvements .// degradation of functionality.
\item
  \texttt{make\ update} is for grabbing the latest version off the
  server-- do this at least at the start of each day.
\item
  \texttt{make\ status} is for finding files that are not currently
  known to Github.
\item
  \texttt{make\ ready} remembers your Github password for one hour-- use
  this if you use \texttt{make\ typo} a lot and you want to save some
  keystrokes.
\item
  \texttt{make\ timm} should be used if Github complains that it does
  not know who you are. Before running this one, edit this rule to
  include your name and email.
\end{itemize}

Of course, there are 1000 other things you can do with a
\texttt{Makefile}. For example, this book is auto-generated by a
\texttt{Makefile} that automatically extracts comments and code from my
Python source code, then compiles the comments as Markdown, then used
the wonderful \texttt{pandoc} tool to compile the Markdown into Latex,
then converts the Latex to a \texttt{.pdf} file. Which is all
interesting stuff-- but beyond the scope of this book.

\subsubsection{Cloud9}\label{cloud9}

If you do not want to install code locally on your machine, then there
are many readily-available on-line integrated development environments.

For example, to have root access to a fully-configured Unix
installation, you can go to

\begin{lstlisting}
 http://c9.io
\end{lstlisting}

One tip is to host your Cloud9 workspace on Github. As of June 2015, the
procedure for doing that was:

\begin{itemize}
\itemsep1pt\parskip0pt\parsep0pt
\item
  Go to Github and create an empty repository.
\item
  Log in to Cloud9 using your GitHub username (at \texttt{http://c9.io},
  there is a button for that, top right).
\item
  Hit the green \emph{CREATE NEW WORKSPACE} button

  \begin{itemize}
  \itemsep1pt\parskip0pt\parsep0pt
  \item
    Select \emph{Clone from URL};
  \item
    Find \emph{Source URL} and enter in
    \texttt{http://github.com/you/yourRepo}
  \item
    Wait ten seconds for the screen to change.
  \item
    Hit the green \emph{START EDITING} button.
  \end{itemize}
\end{itemize}

This will drop you into the wonderful Cloud9 integrated development
environment. Here, is my editting the above Makefile and some Python
code at Cloud9. I've just run \texttt{make\ typo} so all the changes to
the Python file are now backed up outside of Cloud9, over at
\texttt{Github.com}.

\begin{figure}[htbp]
\centering
\includegraphics{img/c9400.png}
\caption{cloud9}
\end{figure}

\subsection{Python101}\label{python101}

\subsubsection{Why Python?}\label{why-python}

I use Python for two reasons: readability and support. Like any computer
scientist, I yearn to use more powerful languages like LISP or
Javascript or Haskell. That said, it has to be said that good looking
Python is reads pretty good-- no ugly brackets, indentation standards
enforced by the compiler, simple keywords, etc.

Ah, you might reply, but what about other beautiful languages like
CoffeeScipt or Scala or insert yourFavoriteLanguageHere? It turns out
that, at the time of this writing, that there is more tutorial support
for Python that any other language I know. Apart from the many excellent
Python textbooks, the on-line community for Python is very active and
very helpful; e.g.~see stackoverlow.com.

\subsubsection{Installing a ``Good'' Python
Environment}\label{installing-a-good-python-environment}

\subsubsection{Python Standards}\label{python-standards}

This textbook uses Python 2.7 for its code base. Of course, it is
tempting to use Python3 but there are still too many Python packages out
there t

\subsection{Homework}\label{homework}

\subsubsection{Homework1}\label{homework1}

\begin{itemize}
\itemsep1pt\parskip0pt\parsep0pt
\item
  Do: get an account at \texttt{http://github.com}. Hand-in: your Github
  id.
\end{itemize}

\input{lib}

\section{Pandoc with citeproc-hs}\label{pandoc-with-citeproc-hs}

\said{item3}

\bibliography{refs.bib}
\end{document}
