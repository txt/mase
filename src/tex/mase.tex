\documentclass[9pt,twocolumn]{article}





\usepackage{graphicx}
\usepackage{natbib}
\bibliographystyle{unsrtnat}
\usepackage{lmodern}
\usepackage{amssymb,amsmath}
\usepackage{ifxetex,ifluatex}
\usepackage{fixltx2e} % provides \textsubscript
\ifnum 0\ifxetex 1\fi\ifluatex 1\fi=0 % if pdftex
  \usepackage[T1]{fontenc}
  \usepackage[utf8]{inputenc}
\else % if luatex or xelatex
  \ifxetex
    \usepackage{mathspec}
    \usepackage{xltxtra,xunicode}
  \else
    \usepackage{fontspec}
  \fi
  \defaultfontfeatures{Mapping=tex-text,Scale=MatchLowercase}
  \newcommand{\euro}{€}
  \fi
  \usepackage{wrapfig}
% use upquote if available, for straight quotes in verbatim environments
\IfFileExists{upquote.sty}{\usepackage{upquote}}{}
% use microtype if available
\IfFileExists{microtype.sty}{%
\usepackage{microtype}
\UseMicrotypeSet[protrusion]{basicmath} % disable protrusion for tt fonts
}{}
\ifxetex
  \usepackage[setpagesize=false, % page size defined by xetex
              unicode=false, % unicode breaks when used with xetex
              xetex]{hyperref}
\else
  \usepackage[unicode=true]{hyperref}
  \fi

%  \usepackage[T1]{fontenc}
\renewcommand{\rmdefault}{ptm}
\usepackage{fourier}

\usepackage[scaled=0.875]{helvet} % ss
\renewcommand{\ttdefault}{pcr} %tt

\usepackage[dvipsnames]{xcolor}
\hypersetup{breaklinks=true,
            bookmarks=true,
            pdfauthor={},
            pdftitle={},
            colorlinks=true,
            citecolor=blue,
            urlcolor=blue,
            linkcolor=magenta,
            pdfborder={0 0 0}}
\urlstyle{same}  % don't use monospace font for urls
\setlength{\parindent}{0pt}
\setlength{\parskip}{6pt plus 2pt minus 1pt}
\setlength{\emergencystretch}{3em}  % prevent overfull lines

\usepackage{lastpage}
\usepackage{fancyhdr}
\pagestyle{fancy}

\rhead{} \chead{}\lhead{}
\cfoot{}

\fancyfoot[LE,RO]{page~\thepage~of~\pageref{LastPage}}
\fancyfoot[LO,RE]{\rotatebox[origin=c]{180}{\copyright} 2015, Tim Menzies, sort of.}



  \renewcommand{\footrulewidth}{0.4pt}

\date{}
%\usepackage{times}
\usepackage[margin=0.75in]{geometry}

\makeatletter
\newcommand{\verbatimfont}[1]{\def\verbatim@font{#1}}%
\makeatother
\usepackage{fancyvrb}



\DefineVerbatimEnvironment%
  {MyVerbatim}{Verbatim}
  {numbers=right,numbersep=2mm,firstnumber=last,stepnumber=1,xleftmargin=12pt,xrightmargin=24pt,
   frame=lines,framerule=0.1mm,fontsize=\scriptsize,rulecolor=\color{Gray}}

  \definecolor{britishracinggreen}{rgb}{0.0, 0.26, 0.15}
  \definecolor{bulgarianrose}{rgb}{0.28, 0.02, 0.03}
  \definecolor{coolblack}{rgb}{0.0, 0.18, 0.39}
  \definecolor{mygreen}{rgb}{0,0.6,0}
  \definecolor{mymauve}{rgb}{0.58,0,0.82}
  
  \usepackage{listings}

  \colorlet{shadecolor}{gray!8}
  \lstset{% general command to set parameter(s)
    language=Python,
    firstnumber=last,
    xleftmargin=6pt,
    xrightmargin=12pt,
    numberblanklines=false,
    backgroundcolor = \color{shadecolor},
    numbers=right, stepnumber=1, numberstyle=\tiny, numbersep=1pt,
    basicstyle=\linespread{0.90}\scriptsize\ttfamily, % print whole listing small
    keywordstyle=\color{coolblack}\bfseries,
    emphstyle=\ttb\color{bulgarianrose},    % Custom highlighting style
    stringstyle=\color{britishracinggreen},
    commentstyle=\color{red},
    stringstyle=\color{mygreen},
    frame=tb,framerule=0.5pt,rulecolor=\color{Gray},
      showstringspaces=false,
    } % no special string spaces

 \newcommand{\said}[1]{\citet*{#1}}


 \hypersetup{linkcolor=black}
\setcounter{tocdepth}{3}

\newcommand{\eq}[1]{Equation~\ref{eq:#1}}
\newcommand{\bi}{\begin{itemize}}
\newcommand{\ei}{\end{itemize}}
\newcommand{\be}{\begin{enumerate}}
\newcommand{\ee}{\end{enumerate}}
\newcommand{\tion}[1]{\textsection\ref{sec:#1}}
\newcommand{\fig}[1]{Figure~\ref{fig:#1}}

\setlength{\columnsep}{5mm}
\setlength{\columnseprule}{0.1pt}

\renewcommand{\headwidth}{7in}
%% \makeatletter
%%     \def\headrule{{\if@fancyplain\let\headrulewidth\plainheadrulewidth\fi
%%         \hrule\@height\headrulewidth\@width 7in \vskip-\headrulewidth}}
%% \makeatother

%% \makeatletter
%%     \def\footrule{{\if@fancyplain\let\headrulewidth\plainheadrulewidth\fi
%%         \hrule\@height\headrulewidth\@width 7in \vskip-\headrulewidth}}
%% \makeatother


\usepackage{moresize}



\begin{document} 

%\renewcommand{\verbatim@font}{\ttfamily\small}

\onecolumn
\title{
  {\bf \HUGE{Evil Code for Wicked Problems, part 4}\\~\\
    \includegraphics[width=6in]{img/herGoggles.png}\\
    \Large{A Research Programmer's Guide to World Domination-- in Python.}\\
    \vspace{-2mm}
    \Large{a.k.a. {\em Lecture Notes, Automated
  SE}, CS, NC State, Fall'15}}}
\author{by Tim Menzies \\\#attentionDeficitSquirrel\\\date{\today}
   }

\maketitle

\begin{center}

 \begin{minipage}{.7\linewidth}
   ~\hrule~

   {\bf SYNOPSIS:} This book is a  ``how to guide'' on model-based reasoning using
  search-based tools (with examples taken from software engineering).\vspace{3mm}
  
  The book builds, from the ground up, numerous
  tiny tools that can tame seemingly complex
  tasks. The tools include methods for
  representing models; reasoning about the
  many goals of those models using state-of-the-art
  algorithms; and discovering new tools are better than
  the state-of-the-art. \vspace{3mm}
  
  
  The audience for this book are graduate  students taking
  a one semester subject in advanced programming methods as
  well as researchers developing the next generation
  of model-based reasoning tools.
   \vspace{3mm}
  
  The code base for the book is very small and written
  in Python 2.7.

  ~\hrule~
  
  \end{minipage}

  \end{center}
\thispagestyle{empty}
\clearpage
\small
\twocolumn

\tableofcontents

\vfill

\section*{Content Advisory }
\begin{wrapfigure}{r}{1.2in}
  \includegraphics[width=1.2in]{img/shark.jpg}
  \end{wrapfigure}
This book contains strong language, weakly typed (and tapped with glee).
This book may contain excessive or gratutious fun--
as well as ideas that some readers may (or may not) find disturbing.
This book does not necessarily 
believed or endorse those ideas- but plays with them anyway
(and asks you to do the same).
This book may include heresies, not suitable for established wisdom. It is
intended for mature audiences only; i.e.  those
old enough to know there is much left to know. 
This book may (or may not) contain peanuts or tree nut products.
Batteries not included.

\newpage

\section*{Source Code Availability and Copyleft}
The software associated with this book
is free and unencumbered and released into the public domain. To download this code, see http://github.com/txt/mase.

Anyone is free to copy, modify, publish, use, compile, sell, or
distribute this software, either in source code form or as a compiled
binary, for any purpose, commercial or non-commercial, and by any
means.

In jurisdictions that recognize copyright laws, the author or authors
of this software dedicate any and all copyright interest in the
software to the public domain. We make this dedication for the benefit
of the public at large and to the detriment of our heirs and
successors. We intend this dedication to be an overt act of
relinquishment in perpetuity of all present and future rights to this
software under copyright law.

THE SOFTWARE IS PROVIDED "AS IS", WITHOUT WARRANTY OF ANY KIND,
EXPRESS OR IMPLIED, INCLUDING BUT NOT LIMITED TO THE WARRANTIES OF
MERCHANTABILITY, FITNESS FOR A PARTICULAR PURPOSE AND NONINFRINGEMENT.
IN NO EVENT SHALL THE AUTHORS BE LIABLE FOR ANY CLAIM, DAMAGES OR
OTHER LIABILITY, WHETHER IN AN ACTION OF CONTRACT, TORT OR OTHERWISE,
ARISING FROM, OUT OF OR IN CONNECTION WITH THE SOFTWARE OR THE USE OR
OTHER DEALINGS IN THE SOFTWARE.

For more information, please refer to http://unlicense.org


\section*{About the Author}

\begin{wrapfigure}{r}{1.5in}
\includegraphics[width=1.5in]{img/tim.png}
\end{wrapfigure}
Tim Menzies (Ph.D., UNSW, 1995, http://menzies.us) is a full Professor in CS at North Carolina State University where he teaches software engineering and automated software engineering. His research relates to synergies between human and artificial intelligence, with particular application to data mining for software engineering.

In his career, he has been a lead researcher on projects for NSF, NIJ, DoD, NASA, USDA, as well as joint research work with private companies.
He is the author of over 230 referred publications; and is one of the 100 most cited authors in software engineering out of over 80,000 researchers.

Prof. Menzies is an associate editor of IEEE
Transactions on Software Engineering, Empirical
Software Engineering and the Automated Software
Engineering Journal. His community service includes
co-founder of the PROMISE project (storing data for repeatable SE
experiments);
co-program chair for the 2012 conference on Automated SE and the 2015
New Ideas and Emerging Research track at the International Conference on SE;
and 
co-general chair for 2016 International Conference on Software Maintenance and Evolution.

Prof. Menzies can be contacted at \verb!tim.menzies@gmail.com!.




\newpage
\onecolumn
\clearpage
\vspace*{\fill}
\begin{center}
  \begin{minipage}{.6\textwidth}
    { \fontsize{450}{450} \selectfont {\bf I}}\hspace{-4mm}\LARGE{ntroduction}
\end{minipage}
\end{center}
\vfill % equivalent to \vspace{\fill}
\clearpage


\fancyhead[LE,RO]{\slshape \rightmark}
\fancyhead[LO,RE]{\slshape \leftmark}

\twocolumn

\section{Welcome to the Evil Plan}\label{welcome-to-the-evil-plan}

\textbf{``The world is a dangerous place to live, not because of the
people who are evil, but because of the people who don't do anything
about it.''} - Albert Einstein

The evil plan (by programmers) to take over the world is progressing
nicely. Certain parts of that plan were initially somewhat undefined.
However, given recent results, this book can now fill in the missing
details from part4 of that plan.

But first, a little history. As all programmers know, the initial parts
of the plan were completed years ago. Part one was was programmers to
adopt a meek and mild persona (possibly even boring and dull).

Part two was, under the guise of that persona, ingratiated ourselves to
government and indistrial agenices (education, mining, manufacturing,
etc etc). Once there, make our work essential to their day to day
opertion. Looking around the world today, it it is plain to see that
part two was very successful.

After that, part three was to make much more material available for our
inspection and manipluation. To this end, the entire planet was enclosed
a digital network- thus giving us unprecendented access to petabytes of
sensors and effectors. Also, by carefully seeding a few promienet
examples of successful programmers (Bill Gates, Steve Jobs, Mark
Zuckerburg), we convinced a lot of people to write lots of little tools,
each of which represent or control some thing, somewhere.

Part four was a little tricky but, as shown in this book, it turned out
not to be too hard. Having access to many models and much data can be
overwhelming-- unless some GREAT SECRET can be used to significantly
simply all that information. For the longest time, that GREAT SECRET was
unknown. However, recent advances have revealed the GREAT that SECRET--
if we describe something in \emph{N} dimensions, then there is usually a
much smaller set of \emph{M} dimensions that contain most of the signal.
The GREAT SECRET is that is it very easy (and fast) to find, then
exploit, those few number of \emph{M} dimensions.

\begin{wrapfigure}{r}{1.3in}
\includegraphics[width=1.3in]{img/evillaugh.jpg}
\end{wrapfigure}

With those controllers in hand, we are now free to move to part five;
i.e.~taking over the world. In fact, the truly evil part of this work is
that know you know you have the power to change the world. Which also
means (evil laugh) you have the guilt if you do not use that power to
right the wrongs of the world. So welcome to a lifetime of discontent
(punctuated by the occasionaly, perhaps fleerting, truimphs) as you
struggle to solve a very large number of pressing problems facing
humanity.

'Nough said. Good luck taking over the world. Remember: if you don't try
then you won't be able to sleep at night, ever again (final evil laugh).

\newpage
\subsection{Research Programming}\label{research-programming}

Silliness aside, this book is about how to be a \emph{research
programmer}. Research programmer's understand the world by:

\begin{itemize}
\itemsep1pt\parskip0pt\parsep0pt
\item
  Codify out current understanding of ``it'' into a model.
\item
  Reasoning about the model.
\end{itemize}

We take this term ``research programmer'' from Ph.D.~Steve Guao's 2012
dissertation.

\subsubsection{Challenges with Research
Programming}\label{challenges-with-research-programming}

Research programming sounds simple, right? Well, there's a catch
(actually, there are several catches).

Firstly, models have to be written and it can be quite a task to create
and validate a model of some complex phenomenon.

see also list in sbse14

Secondly, many models related to \emph{wicked problems};
i.e.\textasciitilde{}problems for which there is no clear best solution.
Tittel XXXWorse still, some models relate to \_wicked there is final
matter of the \emph{goals} that humans want to achieve with those
models. When those goals are contradictory (which happens, all too
often), then our model-based tools must negotiate complex trade offs
between different possibilities.

Thirdly, if wicked problems were not eough, there is also the issue of
uncertainty. Many real world models contain large areas of uncertainty,
especially if that model relates to something that humans have only been
studying for a few decades.

Fourthly, even if you are still not worried about the effectiveness of
reserach problem, consider the complexity of real-world phenomonem. Many
of these models are so complex that we cannot predict what happens when
the parts of that model interact.

Sounds simple, right? Well, there's a catch. Many models related to
\emph{wicked problems}; i.e.~problems for which there is no clear best
solution. Tittel XXXWorse still, some models relate to \_wicked there is
final matter of the \emph{goals} that humans want to achieve with those
models. When those goals are contradictory (which happens, all too
often), then our model-based tools must negotiate complex trade offs
between different possibilities.

If wicked problems were not eough, there is also the issue of
uncertainty. Many real world models contain large areas of uncertainty,
especially if that model relates to something that humans have only been
studying for a few decades.

And if you are still not worried about the effectiveness of reserach
problem, consider the complexity of real-world phenomonem. Many of these
models are so complex that we cannot predict what happens when the parts
of that model interact.

\subsubsection{Parts}\label{parts}

\begin{itemize}
\itemsep1pt\parskip0pt\parsep0pt
\item
  Domain specifc langauges (representation)
\item
  execution (nuktu-objective ootiization)
\item
  evaluation (statistical methods for experimental sciencetists in SE)
\item
  Philophsopy (about what it means to know, and to doubt)
\end{itemize}

\subsubsection{Implications for Software
Engineering}\label{implications-for-software-engineering}

Note that research programming changes the nature and focus and role of
21st century software engineering:

\begin{itemize}
\itemsep1pt\parskip0pt\parsep0pt
\item
  Traditionally, software engineering is about services that meet
  requirements.
\item
  But with research programming, software engineering is less about
  service than about search. Research programming's goal is the
  discovery of interesting features in existing models (or perhaps even
  the evolution of entirely new kinds of models).
\end{itemize}

For example, old-fashioned software engineerings might explore small
things like strings or ``hello world''. But with research programmers
explore \textbf{BIG} things like String Theory or ``hello world model of
climate change and economic impacts''.



\subsection*{The GREAT SECRET}

\input{lib}

\section{Pandoc with citeproc-hs}\label{pandoc-with-citeproc-hs}

\said{item3}

\bibliography{refs.bib}
\end{document}
